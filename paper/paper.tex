\documentclass{amsart}
\usepackage{amsmath}
\usepackage{amssymb}
\usepackage{amsthm}


\title{Experiments with the Maurer-Cartan form}
\author{Finlay Thompson}


\newcommand{\HH}{\mathbb{H}}
\newcommand{\RR}{\mathbb{R}}
\newcommand{\ZZ}{\mathbb{Z}}
\newcommand{\so}{\mathfrak{so}}
\newcommand{\su}{\mathfrak{su}}
\renewcommand{\sp}{\mathfrak{sp}}
\renewcommand{\Re}{\operatorname{Re}}
\newcommand{\Br}{\operatorname{Br}}
\newcommand{\Id}{\operatorname{Id}}
\newcommand{\End}{\operatorname{End}}
\newtheorem{theorem}{Theorem}


\begin{document}


\begin{abstract}
  This paper is a close look at the Maurer-Cartan form in the context of the quaternions.
\end{abstract}

\maketitle

\section{Introduction}

The four dimensional quaternion algebra $\HH$ splits into a one-dimensional and three-dimensional sub-spaces.
If we think of $\HH$ as a Lie algebra, this can be presented as:
$$\HH = \RR \oplus \so(3) = \RR \oplus \su(2) = \RR \oplus \sp(1) $$

We represent the simple three dimensional Lie algebra in one of three different forms, corresponding to real, complex, and quaternionic.

The algebra of quaternions was first described by Irish mathematician Sir William Hamilton in 1843.
Hamilton was so excited about his discovery that he carved the formula into the foot bridge in Dublin.
This has been commemorated by the people of Dublin ever since.

Almost twenty years later, in 1861, Scottish mathematician James Maxwell wrote his electromagnetic equations.
They represented a monumental step forward in physics, and relied on the new vector calculus developed by mathematician Willam Thomson.
Maxwell's equations are commemorated in Edinburgh with a statue of James Maxwell.

Maxwell, in his 1873 "A treatise on electricity and magnetism", makes direct reference to quaternions but does not rely on them to present his equations.
He did make use of $i$, $j$, and $k$ throughout, a notation that Hamilton introduced.

At the end of the 19th century mathematicians studied Maxwell's equations to understand what transformations or symmetries would leave the equations invariant.
The resulting symmetry group was named in 1905 by Poincaré as the Lorentz group.
It was in this form that Einstein was able to interpret them correctly in his theory of special relativity.

While Maxwell's equations were being analysed, a war of words around the use of quaternions in vector calculus was being had in scientific publications.
The main result of this controversy was the side-lining of quaternions for physical applications, and mostly abandoned by mathematicians.

The simple fact that quaternions split into a one-dimensional and three-dimensional parts continues hint at the prospect of understanding the Maxwell's equations in a way that is natural from the point of view of quaternions.

This paper presents presentations of the Lorentz group and Maxwell's electromagnetic equations using quaternions.


\section{The Lorentz Group}

The Lorentz group can be presented as the linear transformations of $\RR^4$ that preserve the Minkowski metric, that has signature $(1,3)$.
The metric is easily presented for $X,Y \in \HH$ as the ``real'' part of the product:
$$ \eta(X,Y) = \Re[XY]$$

If we write $X = X_0 + \mathbf{X}$ and $Y = Y_0 + \mathbf{Y}$ then:
$$ \eta(X,Y) = X_0Y_0 - \mathbf{X}\cdot \mathbf{Y} $$

\subsection{The Brauer isomorphism}

The Brauer group for the real numbers $\RR$ is isomorphic to $\ZZ_2$, and has one generator corresponding to the algebra of quaternions, $\HH$.
This information is equivalent to the existence of an isomorphism:
$$ \Br : \HH\otimes_\RR\HH \to \End_\RR(\HH, \HH) $$

The Brauer isomorphism allows us to express linear maps of $\RR^4$, or four-by-four matricies, as tensor products of quaternions.

\begin{theorem}[Brauer isomorphism]
  For any $X,Y,Z\in\HH$, the map defined by
$$ \Br(X\otimes Y) : Z \mapsto XZY $$
  is an isomorphism of associative algebras $\HH\otimes_\RR\HH \to \End_\RR(\HH,\HH)$\\
  Moreover, $\Br\circ\Br = 4 \cdot \Id$. We call this map the Brauer isomorphism.
\end{theorem}

\begin{proof}

The dimension of both domain and range are equal to $4^2=16$.\\
$\HH$ is a division domain, meaning that
  $$\forall X\in\HH, XY=0 \Rightarrow Y=0$$
This implies that the kernel of $\Br$ is zero.
\end{proof}


\end{document}
